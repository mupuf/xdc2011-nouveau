\documentclass[11pt,english,compress]{beamer}
\usepackage[utf8]{inputenc}
\usepackage{verbatim}
\usepackage{eurosym}
\usepackage{ stmaryrd }

\useoutertheme{smoothbars}
\useinnertheme[shadow=true]{rounded}
\usecolortheme{orchid}
\usecolortheme{whale}
\title{Nouveau}
\subtitle{Recap, on-going and future work}
\author{Karol Herbst, Pierre Moreau \& Martin Peres}
\institute{Nouveau developers}

\AtBeginSection[]{
  \begin{frame}{Summary}
  \small \tableofcontents[currentsection, hideothersubsections]
  \end{frame} 
}

\begin{document}

\setbeamertemplate{navigation symbols}{}

\begin{frame}
	\titlepage
\end{frame}

\section{Introduction}
\subsection*{Introduction}
\begin{frame}
	\frametitle{Introduction}

	\begin{block}{Introduction}
		\begin{itemize}
			\item Nouveau is Linux's OSS driver for NVIDIA GPUs;
			\item We want to provide a good out-of-the-box desktop experience;
			\item We wish to run games and compute workloads too!
		\end{itemize}
	\end{block}

	\pause

	\begin{block}{Support}
		\begin{itemize}
			\item All NVIDIA desktop GPUs since 1998 (partial support);
			\item 2D and 3D acceleration on all recent GPUs (2003+);
			\item OpenGL 4.4 (non official) \& Direct X 9 (through Wine);
			\item Video decoding on GPUs between 2004 and 2013.
		\end{itemize}
	\end{block}
\end{frame}

\section{Pascal support}
\subsection*{Pascal support}
\begin{frame}
	\frametitle{Pascal support}

	\begin{block}{Pascal GPUs}
		\begin{itemize}
			\item Current generation of GPUs, released in March 2016;
			\item Most locked-up NVIDIA GPUs to date;
			\pause
			\item Supported features:
			\begin{itemize}
				\item Modesetting: Complete support (Linux 4.14);
				\item 2D/3D acceleration: Yes, but firmwares came after 1+ year;
				\item Temperature reading: Yes;
				\pause
				\item Fan management: Impossible (locked down);
				\item Reclocking: Impossible (locked down);
				\item Video BIOS uploading: Impossible (locked down);
				\item Power reading: Impossible (locked down).
			\end{itemize}
		\end{itemize}
	\end{block}
\end{frame}

% Martin
\subsection{Power Management}

\begin{frame}
	\frametitle{Power Management}

	\begin{block}{Clock gating (Lyude Paul)}
		\begin{itemize}
			\item Increases the battery life of laptops without performance loss;
			\item Experimental version for Kepler about to land in Nouveau.
		\end{itemize}
	\end{block}

	\begin{block}{Fan management (Martin Peres)}
		\begin{itemize}
			\item Adjusts the fan speed based on the temperature;
			\item Full support for most GPUs since Linux 3.13;
			\item Some GPUs require a weird calibration: Loud fans!
			\item NVIDIA is about to release some documentation for this.
		\end{itemize}
	\end{block}
\end{frame}

\begin{frame}
	\frametitle{Power Management}

	\begin{block}{Reclocking (Karol Herbst \& Roy Spliet)}
		\begin{itemize}
			\item WIP Karol
		\end{itemize}
	\end{block}

	\begin{block}{Power monitoring}
		\begin{itemize}
			\item Power consumption exposed when available;
		\end{itemize}
	\end{block}
\end{frame}

\section{Userspace}
\subsection{OpenGL}

\begin{frame}
	\frametitle{OpenGL}

% 	TODO: Karol

	\begin{block}{History: GL version support}
		\begin{itemize}
			\item OpenGL 3.0 in Mesa 8.0 (nvc0);
			\item OpenGL 3.1 in Mesa 9 (nvc0);
			\item OpenGL 3.3 support in Mesa 10.1 for nv50/nvc0.
		\end{itemize}
	\end{block}

	\begin{block}{Limited support}
		\begin{itemize}
			\item GK110;
			\item GK208.
		\end{itemize}
	\end{block}
\end{frame}

\subsection{Direct 3D}

\begin{frame}
	\frametitle{Nine: a d3d9 state tracker}

	\begin{block}{Nine: a d3d9 state tracker}
		\begin{itemize}
			\item Started by Joakim Sindholt;
			\item Completed by Christoph Bumiller
			\item Runs Skyrim, Civilization 5, Anno 1404 and StarCraft 2;
			\item Up to 2 times faster than Wine's d3d implementation.
		\end{itemize}
	\end{block}

	\begin{block}{Announcement}
		\url{http://lists.freedesktop.org/archives/mesa-dev/2013-July/041900.html}
	\end{block}

	\begin{block}{Source tree}
		\url{https://github.com/chrisbmr/Mesa-3D/tree/gallium-nine}
	\end{block}
\end{frame}

\subsection{OpenCL}
\begin{frame}
	\frametitle{OpenCL}

	% Small intro/history of OpenCL ~30sec
	% SPIR-V: quick summary, why interesting? ~1min
	% modifications to clover, for SPIR-V, affects everyone 2min
	% how to try it ~30sec
	% quick overview on current SPIR-V -> NVIR work ~1-2min
	% how to try it? ~30sec
	% Total: ~6-7min
	\begin{block}{}
	\end{block}
% 	TODO: Pmoreau

\end{frame}

\begin{frame}{Try out the SPIR-V support}
	\begin{block}{Prerequisites}
		\begin{itemize}
			\item SPIRV-Tools:
				\url{https://github.com/KhronosGroup/SPIRV-Tools}
			\item llvm-spirv:
				\url{https://gitlab.collabora.com/tomeu/llvm-spirv}
			\item LLVM $\geq 5.0$
			\item Mesa: \url{https://github.com/pierremoreau/mesa}
				(branch: clover\_spirv\_series\_v3)
		\end{itemize}
	\end{block}
	\begin{block}{How to use/test it?}
		\begin{itemize}
			\item Set CLOVER\_USE\_SPIRV=1;
			\item Use clCreateProgramWithILKHR(), clCreateProgramWithIL();
			\item Or for AMD owners, use clCreateProgramWithSource().
		\end{itemize}
	\end{block}
\end{frame}

\begin{frame}{Overview of SPIR-V to NVIR}
	\begin{block}{Status for OpenCL 1.2 support}
		\begin{itemize}
			\item Supported:
			\begin{itemize}
				\item Most arithmetic/relational/bit/etc. operations;
				\item Most atomics and convert operations;
				\item Function calling and control flow.
			\end{itemize}
			\item Work in progress:
			\begin{itemize}
				\item Image support;
				\item Finishing off the various memory operations.
			\end{itemize}
			\item Still missing:
			\begin{itemize}
				\item Group operations;
				\item Most OpenCL specific operations.
			\end{itemize}
		\end{itemize}

		OpenCL CTS passing rates for test\_basic: 36/95 (27 of the
		failing ones are image tests)
	\end{block}
\end{frame}

\begin{frame}{Trying out OpenCL on Nouveau}
	\begin{block}{Prerequisites}
		\begin{itemize}
			\item Same as for testing the SPIR-V support;
			\item except for Mesa, for which the branch is
				nouveau\_spirv\_support.
		\end{itemize}
	\end{block}
	\begin{block}{Hardware status}
		\begin{itemize}
			\item Tesla: needs changes to the memory management
				code;
			\item Fermi: should work;
			\item Kepler: should work;
			\item Maxwell: partially works;
			\item Pascal: untested;
			\item Volta: needs reverse engineering for Linux and
				Mesa support.
		\end{itemize}
	\end{block}
\end{frame}

% Xexaxo
\section{Community}

\subsection{History with NVIDIA}
\begin{frame}
	\frametitle{Community - Welcome NVIDIA!}

% 	TODO: Martin

	\begin{block}{Flash news}
		\begin{itemize}
			\item NVIDIA released NDA-free documentation during
XDC2013!;
			\item Offered us a contact email to answer questions;
			\item Are willing to improve the out-of-the-box
experience of users;
			\item Provided documentation on the DCB-related vbios
tables;
			\item Helped us get MSI IRQs working and fix video
decoding;
			\item Released a GPL Tegra K1 driver with extensive reg
dumps;
			\item Sent an RFC to support the Tegra K1 driver in Nouveau;
			\item Welcome to the Nouveau community, NVIDIA!
		\end{itemize}
	\end{block}
\end{frame}

\end{document}
