\documentclass[11pt,english,compress]{beamer}
\usepackage[utf8]{inputenc}
\usepackage{verbatim}
\usepackage{eurosym}
\usepackage{ stmaryrd }

\useoutertheme{smoothbars}
\useinnertheme[shadow=true]{rounded}
\usecolortheme{orchid}
\usecolortheme{whale}
\title{Nouveau}
\subtitle{The community \& past, current and future developments}
\author{Martin Peres \& the Nouveau community}
\institute{Ph.D. student at LaBRI}
%\logo{\includegraphics[width=1.5cm]{imgs/ensib.jpg}}

\AtBeginSection[]{
  \begin{frame}{Summary}
  \small \tableofcontents[currentsection, hideothersubsections]
  \end{frame} 
}

\begin{document}

\setbeamertemplate{navigation symbols}{}

\begin{frame}
	\titlepage
\end{frame}

\section{History}
	\begin{frame}
		\begin{block}{History: NVIDIA, a new hope}
			\begin{itemize}
				\item 1998(?): NVIDIA releases a Linux open-source 2D-only driver(nv)
				\item 1998: Obfuscation commit (release only pre-processed source)
			\end{itemize}
		\end{block}
	\end{frame}

	\begin{frame}
		\begin{block}{}
			After we already finalized XFree86-3.3.3 NVIDIA forced The XFree86 Project
			to replace the sources we had with sources that were partly run through the
			C preprocessor in order to remove some of the names that NVIDIA thought
			might give away IP from NVIDIA. This resulted in unreadable and unmaintainable
			code.
		\end{block}

		\begin{block}{}
			The XFree86 Project is strongly opposed to such obfuscated code. We do not
			regard this as free software according to our standards. Due to the extremely
			late date of this decision from NVIDIA we decided to include the code as
			offered by NVIDIA. We are considering to remove support for the later NVIDIA
			chips in a future release, though.
		\end{block}

		 Xfree86, commit message from 11/18/98
% http://cvsweb.xfree86.org/cvsweb/xc/programs/Xserver/hw/xfree86/vga256/drivers/nv/Attic/README.RIVATNT.diff?r1=1.1.2.2&r2=1.1.2.3&hideattic=0&only_with_tag=xf-3_3_3
% http://cvsweb.xfree86.org/cvsweb/xc/programs/Xserver/hw/xfree86/vga256/drivers/nv/Attic/nv3driver.c.diff?r1=1.1.2.5&r2=1.1.2.6&hideattic=0&only_with_tag=xf-3_3_3
	\end{frame}

	\begin{frame}
		\begin{block}{History: The open-source strikes back}
			\begin{itemize}
				\item 2005: Stephane Marchesin improves nv and works on 3D
				\begin{itemize}
					\item Project named Nouveau after an unfortunate automatic spelling correction
					\item Back-port nv updates to Nouveau
				\end{itemize}
				\item 2008: Open Arena runs on nv40
				\item 2009: KMS driver based on TTM
				\item 2010: Merged in Linux 2.6.33
				\item 2010: Nv is deprecated by NVIDIA, say ``use VESA''.
			\end{itemize}
		\end{block}
	\end{frame}

	\begin{frame}
		\begin{block}{History: The return of the Jedi}
			``It's so hard to write a graphics driver that open-sourcing it
			would not help [...] In addition, customers aren't asking for opensource drivers.''\\
			Andrew Fear, NVIDIA software product manager, April 2006
			%http://news.cnet.com/New-Linux-look-fuels-old-debate/2100-7344_3-6061491.html
		\end{block}
	\end{frame}

\section{Architecture}
	\subsection{Chipset families}
		\begin{frame}
			\begin{block}{Chipset families}
				\begin{itemize}
					\item NV03: RIVA 128
					\item NV04: RIVA TNT/TNT2
					\item NV10: Geforce2/4 MX
					\item NV20: Geforce 3, Geforce 4 Ti
					\item NV30: Geforce 5/FX
					\item NV40: Geforce 6/7 (the first real family) 
					\item NV50: Geforce 8/9/100/200/300 (the biggest family)
					\item NVC0: Geforce 400/500, AKA Fermi
					\item NVD9: Half-Kepler chipset
				\end{itemize}
			\end{block}
		\end{frame}

	\subsection{Components}
		\begin{frame}
			\begin{block}{Linux module Nouveau}
				\begin{itemize}
					\item Kernel Mode Setting
					\item Command-submission
					\item Resource allocation
				\end{itemize}
			\end{block}

			\begin{block}{Nouveau DDX}
				\begin{itemize}
					\item EXA (2D acceleration) 
					\item X-Video
				\end{itemize}
			\end{block}

			\begin{block}{Mesa: 3D acceleration}
				\begin{itemize}
					\item Nouveau\_vieux: 3D for NV04,NV10,NV20 (mesa classic)
					\item NVFX: 3D for the NV30,40 families (gallium)
					\item Nouveau: 3D for NV50,C0 families (gallium)
				\end{itemize}
			\end{block}
		\end{frame}

		\begin{frame}
			\begin{block}{NV30/40 microcodes}
				\begin{itemize}
					\item HWSQ: very limited use (LVDS), no flow control
					\item CtxProgs on nv40: Context switching
				\end{itemize}
			\end{block}

			\begin{block}{NV50 microcodes}
				\begin{itemize}
					\item CtxProgs: Context-switching microcode
					\item HWSQ v2 (formerly PMS): memory reclocking
					\item nv98+ Flexible MicroCode(F$\mu$c): a general-purpose microcode. Broad usage (PCrypt, PDaemon, Vdec, ...)
				\end{itemize}
			\end{block}

			\begin{block}{NVC0-D9}
				\begin{itemize}
					\item nvc0: PGRAPH (the rendering engine) is converted to F$\mu$c
					\item nvd9: some remaining engines are converted to F$\mu$c
				\end{itemize}
			\end{block}
		\end{frame}

\section{Past \& current work}
	\subsection{Modesetting, 2D}
		\begin{frame}
			\begin{block}{ModeSetting - Done}
				\begin{itemize}
					\item NV04: TNT2
					\item NV10: Geforce2/4 MX
					\item NV20: Geforce 3, Geforce 4 Ti
					\item NV30: Geforce 5/FX
					\item NV40: Geforce 6/7
					\item NV50: Geforce 8/9/100/200/300 (the biggest family)
					\item NVC0: Geforce 400/500, AKA Fermi
				\end{itemize}
			\end{block}

			\begin{block}{ModeSetting - WIP}
				\begin{itemize}
					\item NVD9: Partially Kepler
				\end{itemize}
			\end{block}
		\end{frame}

	\subsection{3D support}
		\begin{frame}
			\begin{block}{3D Drivers and support}
				\begin{itemize}
					\item nouveau\_vieux (NV04,10,20): classic mesa driver. Unsupported.
					\item nvfx (NV30,40): gallium driver. Works but no maintainer.
					\item nouveau (NV50,C0): gallium driver. Supported! 
				\end{itemize}
			\end{block}
	
			\begin{block}{WIP}
				\begin{itemize}
					\item add nvc1 support
					\item improve performance
					\item crashes with Unigine Tropics \& Heaven on some chipsets
				\end{itemize}
			\end{block}
		\end{frame}

	\subsection{Power management}
		\begin{frame}
			\begin{block}{Power management}
 				\begin{itemize}
					\item Readings: Temperature \& clocks
					\item vbios parsing: mostly
					\item setting clocks: unreliable, potentially dangerous
				\end{itemize}
			\end{block}

			\begin{block}{WIP}
				\begin{itemize}
					\item nvc0: setting clocks
					\item nv40-d9: memory timings (almost ready)
					\item nv30-c0: reliable clock changes (almost ready)
					\item AGP/PCIE, clock gating: reverse engineering
					\item performance counters, dynamic reclocking: WIP
				\end{itemize}
			\end{block}
		\end{frame}

	\subsection{HW video decoding}
		\begin{frame}
			\begin{block}{HW video decoding}
				\begin{itemize}
					\item MPEG1/2: nv40-98
				\end{itemize}
			\end{block}

			\begin{block}{Why is MPEG4 hw decoding so hard?}
				\begin{itemize}
					\item involves at least 4 engines
					\item different ISAs (F$\mu$c, VPx)
					\item codecs needs to be implemented
				\end{itemize}
			\end{block}
		\end{frame}

\section{The Nouveau community}
	\subsection{Composition}
		\begin{frame}
			\begin{block}{The nouveau community}
				\begin{itemize}
					\item The largest xorg-related IRC channel
					\item Composed of:
						\begin{itemize}
							\item One paid developer
							\item Former developers
							\item Student developers
							\item Enthusiasts
						\end{itemize}
				\end{itemize}
			\end{block}

			\begin{block}{The nouveau maintainer}
				Nouveau is maintained by Ben Skeggs(darktama):
				\begin{itemize}
					\item Hired by Red Hat in 2009
					\item Located in Brisbane, Australia (GMT+10)
					\item Works on almost everything
				\end{itemize}
			\end{block}
		\end{frame}

		\begin{frame}
			\begin{block}{Gallium's nouveau maintainer}
				Christoph Bumiller (calim) maintains the nv50-c0 gallium driver:
				\begin{itemize}
					\item Physics master student at the University of Vienna (Austria, GMT+1)
					\item Main Nouveau Gallium contributor
					\item nv50/c0 3D support
					\item performance improvements
				\end{itemize}
			\end{block}

			\begin{block}{Marcin Kościelnicki (mwk)}
				\begin{itemize}
					\item Polish master student at the university of Warsaw (GMT+1)
					\item Implemented most of the GPGPU-oriented PSCNV nouveau fork
					\item Reversed most of the Fermi's architecture and video decoding
				\end{itemize}
			\end{block}
		\end{frame}

		\begin{frame}
			\begin{block}{Pekka Paalanen (pq)}
				\begin{itemize}
					\item Finnish (GMT+2)
					\item Worked on mmiotraces, a register DB
					\item Does some communication-related work
				\end{itemize}
			\end{block}

			\begin{block}{Francisco Jerez (curro)}
				\begin{itemize}
					\item Spanish Physics student (GMT+1)
					\item Worked on page-flipping and nouveau\_vieux
				\end{itemize}
			\end{block}

			\begin{block}{Marcin Slusarz (joi)}
				\begin{itemize}
					\item Polish SQL/C++/Java developer (GMT+1)
					\item Fixes software bugs (mesa + kernel), maintains Valgrind-MMT
				\end{itemize}
			\end{block}
		\end{frame}

		\begin{frame}
			\begin{block}{Emil Velikov (xexaxo)}
				\begin{itemize}
					\item Bulgarian student at the University of Nottingham (England, GMT+0)
					\item Reverse engineering of some PM-related vbios table
					\item Other PM-related implementation work
					\item Debugging/Testing
				\end{itemize}
			\end{block}

			\begin{block}{Roy Spliet (RSpliet)}
				\begin{itemize}
					\item Dutch master student at the Delft University (GMT+1)
					\item Memory timings reverse engineering
				\end{itemize}
			\end{block}
		\end{frame}

		\begin{frame}
			\begin{block}{Martin Peres (mupuf)}
				\begin{itemize}
					\item Engineer, Ph.D. student at LaBRI (France, GMT+1)
					\item Reclocking process
					\item Thermal-zones \& thermal management
					\item Minor reverse engineering
				\end{itemize}
			\end{block}

			\begin{block}{Maxim Levitsky (MaximLevitsky)}
				\begin{itemize}
					\item Student at the Technion University of Haifa (GMT+2)
					\item Important reverse engineering work on reducing power consumption
					\item Stability-related reverse engineering
					\item New-comer
				\end{itemize}
			\end{block}
		\end{frame}

		\begin{frame}
			\begin{block}{Maarten Lankhorst (mlankhorst)}
				\begin{itemize}
					\item Netherlands (GMT+1)
					\item Implemented XVMC support
					\item Work towards an open VDPAU
					\item New-comer
				\end{itemize}
			\end{block}
		\end{frame}

\section{Demos}
	\begin{frame}
		\begin{block}{MPEG1/2 video decoding}
			\begin{itemize}
				\item using MPlayer
				\item Kernel-side: done
				\item Mesa: merged 3 days ago
			\end{itemize}
		\end{block}
	\end{frame}

	\begin{frame}
		\begin{block}{Dynamic reclocking}
			\begin{itemize}
				\item see clocks changing according to the load
				\item performance improvements in OpenArena
			\end{itemize}
		\end{block}
	\end{frame}
\end{document}
