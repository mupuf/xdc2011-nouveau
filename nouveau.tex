\documentclass[11pt,english,compress]{beamer}
\usepackage[utf8]{inputenc}
\usepackage{verbatim}
\usepackage{eurosym}
\usepackage{ stmaryrd }

\useoutertheme{smoothbars}
\useinnertheme[shadow=true]{rounded}
\usecolortheme{orchid}
\usecolortheme{whale}
\title{Nouveau}
\subtitle{Recap, on-going and future work}
\author{Karol Herbst, Pierre Moreau \& Martin Peres}
\institute{Nouveau developers}

\AtBeginSection[]{
  \begin{frame}{Summary}
  \small \tableofcontents[currentsection, hideothersubsections]
  \end{frame} 
}

\begin{document}

\setbeamertemplate{navigation symbols}{}

\begin{frame}
	\titlepage
\end{frame}

\section{Introduction}
\subsection*{Introduction}
\begin{frame}
	\frametitle{Introduction}

	\begin{block}{Introduction}
		\begin{itemize}
			\item Nouveau is Linux's OSS driver for NVIDIA GPUs;
			\item We want to provide a good out-of-the-box desktop experience;
			\item We wish to run games and compute workloads too!
		\end{itemize}
	\end{block}

	\pause

	\begin{block}{Support}
		\begin{itemize}
			\item All NVIDIA desktop GPUs since 1998 (partial support);
			\item 2D and 3D acceleration on all recent GPUs (2003+);
			\item OpenGL 4.5 (non official) \& Direct X 9 (through Wine);
			\item Video decoding on GPUs between 2004 and 2013.
		\end{itemize}
	\end{block}
\end{frame}

\section{Pascal support}
\subsection*{Pascal support}
\begin{frame}
	\frametitle{Pascal support}

	\begin{block}{Pascal GPUs}
		\begin{itemize}
			\item Current generation of GPUs, released in March 2016;
			\item Most locked-up NVIDIA GPUs to date;
			\pause
			\item Supported features:
			\begin{itemize}
				\item Modesetting: Complete support (Linux 4.14);
				\item 2D/3D acceleration: Yes, but firmwares came after 1+ year;
				\item Temperature reading: Yes;
				\pause
				\item Fan management: Impossible (locked down);
				\item Reclocking: Impossible (locked down);
				\item Video BIOS uploading: Impossible (locked down);
				\item Power reading: Impossible (locked down).
			\end{itemize}
		\end{itemize}
	\end{block}
\end{frame}

% Martin
\subsection{Power Management}

\begin{frame}
	\frametitle{Power Management}

	\begin{block}{Clock gating (Lyude Paul)}
		\begin{itemize}
			\item Increases the battery life of laptops without performance loss;
			\item Experimental version for Kepler about to land in Nouveau.
		\end{itemize}
	\end{block}

	\begin{block}{Fan management (Martin Peres)}
		\begin{itemize}
			\item Adjusts the fan speed based on the temperature;
			\item Full support for most GPUs since Linux 3.13;
			\item Some GPUs require a weird calibration: Loud fans!
			\item NVIDIA is about to release some documentation for this.
		\end{itemize}
	\end{block}
\end{frame}

\begin{frame}
	\frametitle{Power Management}

	\begin{block}{Reclocking (Karol Herbst \& Roy Spliet)}
		\begin{itemize}
			\item Move to a partial reclock approach for faster clock updates
			\item Fixing issues related to runpm
			\item Update clock state on temperature changes (needed for dynamic reclocking)
			\item Add support for Engine PMU counters
			\item Fermi reclocking improvements coming
		\end{itemize}
	\end{block}

	\begin{block}{Power monitoring}
		\begin{itemize}
			\item Power consumption exposed when available;
			\item Able to get the power budget on a few cards
		\end{itemize}
	\end{block}
\end{frame}

\section{Userspace}
\subsection{Graphics}

\begin{frame}
	\frametitle{Graphics}

	\begin{block}{History: GL version support (NVC0)}
		\begin{itemize}
			\item OpenGL 4.1 support in Mesa 11.0
			\item OpenGL 4.3 support in Mesa 12.0
			\item OpenGL 4.5 support in Mesa 13.0 (inofficial)
		\end{itemize}
	\end{block}
	
	\pause
	
	\begin{block}{Vulkan}
		\begin{itemize}
			\item NIR to NVIR started for Vulkan SPIR-V support
			\item Also helps for OpenGL 4.6 (ARB\_gl\_spirv and ARB\_spirv\_extensions)
			\item Hopefully some basic Vulkan driver ready this year
		\end{itemize}
	\end{block}
\end{frame}

\subsection{OpenCL}
\begin{frame}
	\frametitle{OpenCL}

	% Small intro/history of OpenCL ~30sec
	% SPIR-V: quick summary, why interesting? ~1min
	% modifications to clover, for SPIR-V, affects everyone 2min
	% X how to try it ~30sec
	% X quick overview on current SPIR-V -> NVIR work ~1-2min
	% X how to try it? ~30sec
	% Total: ~6-7min
	\begin{block}{}
	\end{block}
% 	TODO: Pmoreau

\end{frame}

\begin{frame}{Quick overview of OpenCL}
	% History, goal?, current support by platforms
	% Short example of OpenCL kernel
\end{frame}

\begin{frame}{Quick overview of SPIR-V}
	% History, goal, use, etc.
	% Short example
\end{frame}

\begin{frame}{Adding SPIR-V support to clover}
	% * Two new entrypoints for creating a program: OpenCL >+ 2.1, and
	%   extension for 1.2 and 2.0
	% * Previously, compile to device specified IR.
	% * But now, issue for linking: LLVM IR/NIR/TGSI to link with SPIR-V
	% * Solution, canonical IR: SPIR-V, translate to device specific IR
	%   after creating executable
	\begin{block}{}
	\end{block}
\end{frame}

\begin{frame}{Try out the SPIR-V support}
	\begin{block}{Prerequisites}
		\begin{itemize}
			\item SPIRV-Tools:
				\url{https://github.com/KhronosGroup/SPIRV-Tools}
			\item llvm-spirv:
				\url{https://gitlab.collabora.com/tomeu/llvm-spirv}
			\item LLVM $\geq 5.0$
			\item Mesa: \url{https://github.com/pierremoreau/mesa}
				(branch: clover\_spirv\_series\_v3)
		\end{itemize}
	\end{block}
	\begin{block}{How to use/test it?}
		\begin{itemize}
			\item Set CLOVER\_USE\_SPIRV=1;
			\item Use clCreateProgramWithILKHR(), clCreateProgramWithIL();
			\item Or for AMD owners, use clCreateProgramWithSource().
		\end{itemize}
	\end{block}
\end{frame}

\begin{frame}{Overview of SPIR-V to NVIR}
	\begin{block}{Status for OpenCL 1.2 support}
		\begin{itemize}
			\item Supported:
			\begin{itemize}
				\item Most arithmetic/relational/bit/etc. operations;
				\item Most atomics and convert operations;
				\item Function calling and control flow.
			\end{itemize}
			\item Work in progress:
			\begin{itemize}
				\item Image support;
				\item Finishing off the various memory operations.
			\end{itemize}
			\item Still missing:
			\begin{itemize}
				\item Group operations;
				\item Most OpenCL specific operations.
			\end{itemize}
		\end{itemize}

		OpenCL CTS passing rates for test\_basic: 36/95 (27 of the
		failing ones are image tests)
	\end{block}
\end{frame}

\begin{frame}{Try out OpenCL on Nouveau}
	\begin{block}{Prerequisites}
		\begin{itemize}
			\item Same as for testing the SPIR-V support;
			\item except for Mesa, for which the branch is
				nouveau\_spirv\_support.
		\end{itemize}
	\end{block}
	\begin{block}{Hardware status}
		\begin{itemize}
			\item Tesla: needs changes to the memory management
				code;
			\item Fermi: should work;
			\item Kepler: should work;
			\item Maxwell: partially works;
			\item Pascal: partially works;
			\item Volta: needs reverse engineering for Linux and
				Mesa support.
		\end{itemize}
	\end{block}
\end{frame}

\section{Community}

\subsection{Current members}
\begin{frame}
	\frametitle{Community - members}

	\begin{block}{Red Hat developers working on Nouveau}
		\begin{itemize}
			\item Ben Skeggs: maintainer and long time contributor;
			\item Lyude Paul: part time on power management;
			\item Karol Herbst: full time on reclocking, mesa and Compute.
		\end{itemize}
	\end{block}

	\pause

	% TODO
	\begin{block}{Community}
		\begin{itemize}
			\item Ilia Mirkin: Mesa developer, OpenGL;
			\item Martin Peres: Fan management, power management;
			\item Roy Spliet: Fermi reclocking, compiler opts;
			\item You? Join us!
		\end{itemize}
	\end{block}
\end{frame}

\subsection{History with NVIDIA}
\begin{frame}
	\frametitle{Community - Relationship with NVIDIA}

	\begin{block}{Recent History with NVIDIA}
		\begin{itemize}
			\item Sept 2013: First real contact
			\begin{itemize}
				\item NVIDIA released public vbios documentation (DCB);
				\item Offered us a contact email to answer questions;
				\item Are willing to improve the out-of-the-box
experience of users;
			\end{itemize}
			\item 2015-2017: NVIDIA hired someone to work on Nouveau
			\begin{itemize}
				\item Added Tegra K1/K2 support to Nouveau;
				\item Led to a Nouveau-based product (Pixel-C);
				\item Wrote secure-boot support for Maxwell+;
			\end{itemize}
			\item 2018: New documentation dump for the vbios tables.
		\end{itemize}
	\end{block}
\end{frame}

\begin{frame}
	\frametitle{Community - What we need from NVIDIA}

	\begin{block}{Locked GPUs affect development and user experience}
		\begin{itemize}
			\item 2015: The Maxwell 2+ GPU are locked, signed FWs prevent:
			\begin{itemize}
				\item Accelerated graphics: usually given a year after release;
				\item Fan management: no FW provided;
				\item Reclocking: no FW provided;
				\item Power reading: no FW provided;
				\item VBIOS reverse engineering.
			\end{itemize}
			\item 2018: Some VBIOS documentation landed:
			\begin{itemize}
				\item A website appeared to sign some vbios;
				\item Some signs of opening?
			\end{itemize}
		\end{itemize}
	\end{block}
\end{frame}

\section{Conclusion}

\subsection*{Conclusion}
\begin{frame}
	\frametitle{Conclusion}

	\begin{block}{Nouveau is improving}
		\begin{itemize}
			\item Nouveau is still the default driver in all distributions;
			\item The GL driver is in good shape: OpenGL 4.4 and 4.5 coming;
			\item Performance needs to improve for 4K displays: reclocking!;
			\item Power efficiency for laptop users need to be improved too.
		\end{itemize}
	\end{block}

	\begin{block}{Join the fun?}
		\begin{itemize}
			\item Why not join the team? We have lots of challenges!
			\item GSoC/EVoC students: we'll have projects for you!
		\end{itemize}
	\end{block}
\end{frame}


\end{document}
