\documentclass{beamer}
\usepackage[utf8]{inputenc}
\usepackage[english]{babel}
\usetheme{Antibes}

% À chaque nouvelle section, on dit où on est
\AtBeginSection[ ]
{
 \begin{frame}<beamer>
   \frametitle{Summary}
   \tableofcontents[currentsection]
  \end{frame}
} 

\title{State of the X.Org Foundation}
\subtitle{FOSDEM 2014}
\author{Martin~Peres \& others of the board of directors}
\institute{\includegraphics[height=1.0cm]{figures/logo.png}}

\begin{document}
	\begin{frame}
		\titlepage
	\end{frame}

	\logo{\includegraphics[height=1.0cm]{figures/logo_small_200px.png}}

	\section{Introduction}
		\subsection{Role of the Foundation}
		\begin{frame}
			\begin{block}{}
				The X.Org Foundation is a non-profit corporation chartered to
develop and execute effective strategies that provide worldwide stewardship and
encouragement of the X Window System and related projects (Mesa, DRI, Wayland,
etc.).
			\end{block}

			\begin{block}{}
				\url{http://www.x.org/wiki/XorgFoundation/}
			\end{block}

			\begin{center}
				\includegraphics[width=.15\linewidth]{figures/logo_small_200px.png}
				\hspace{.05\linewidth}
				\includegraphics[width=.15\linewidth]{figures/xcb_logo_small.png}
				\hspace{.05\linewidth}
				\includegraphics[width=.15\linewidth]{figures/tux_small.jpg}
				\hspace{.05\linewidth}
				\includegraphics[width=.15\linewidth]{figures/glxgears-trans.png}
				\hspace{.05\linewidth}
				\includegraphics[width=.15\linewidth]{figures/wayland_logo_small.png}
			\end{center}

		\end{frame}

		\begin{frame}
			\begin{block}{What the Foundation DOESN'T provide}
				\begin{itemize}
					\item technical guidance;
					\item roadmaps/deadlines;
					\item releases;
					\item supervision of any kind.
				\end{itemize}
			\end{block}
			
			\begin{block}{What the Foundation does provide}
				\begin{itemize}
					\item communication tools (in relation with Freedesktop);
					\item an annual physical meeting;
					\item money to help developing the Graphics stack.
				\end{itemize}
			\end{block}
		\end{frame}

		\subsection{Actions of the Foundation}
		\begin{frame}
			\begin{block}{X.Org Developer's Conference (XDC/XDS)}
				\begin{itemize}
					\item Organized once a year, around
September/October;
					\item Alternates between the US and
Europe (in Europe this year!);
					\item Lasts 3 days;
				\end{itemize}
			\end{block}

			\begin{block}{Travel sponsorship}
				The board of directors can cover the travel and
accomodation expenses to the developers who couldn't attend an X.Org-related
conference (XDC or FOSDEM?) otherwise.
			\end{block}
		\end{frame}

		\begin{frame}
			\begin{block}{Google Summer of Code (GSoC)}
				\begin{itemize}
					\item The X.Org Foundation is an
approved org. of the GSoC;
					\item This allows students to participate to
X.Org-related projects over the summer and get paid to flip bits, not burgers.
				\end{itemize}
			\end{block}

			\begin{block}{GSoC projects of 2013}
				\begin{itemize}
					\item David Hermann (Dave Airlie):
Render Nodes and security fixes
					\item Samuel Pitoiset (Martin Peres):
Reverse engineering performance counters on NVIDIA cards and exposing Fermi MP
counters in Mesa;
					\item Dylan Noblesmith (Ian Romanick):
Implementing GL\_EXT\_direct\_state\_access.
				\end{itemize}
			\end{block}
		\end{frame}

		\begin{frame}
			\begin{block}{Endless Vacation of Code (EVoC)}
				\begin{itemize}
					\item EVoC is a GSoC-like project, funded by the X.Org Foundation;
					\item It allows students to participate to
X.Org-related projects during their vacation, at any time of the year.
				\end{itemize}
			\end{block}

			\begin{block}{EVoC projects}
				\begin{itemize}
					\item 2013: No projects;
					\item 2012:
					\begin{itemize}
						\item Francisco Jerrez (None): Gallium/Nouveau - OpenCL support
						\item Supreet Pal Singh (Martin Peres): Nouveau - Software scripting Engine for Fermi Architecture based GPUs;
						\item Blaž Tomažič (?): OpenCL Testing Framework for Piglit. Project Overview 
						\item And more...
					\end{itemize}
				\end{itemize}
			\end{block}
		\end{frame}

		\begin{frame}
			\begin{block}{Communication: Google+ and Twitter}
				\begin{itemize}
					\item Google+/Youtube: Store and/or link
to talk videos, slides, blog articles or G+ posts related to projects under our umbrella (Martin Peres);
					\item Twitter: Mostly security issues, random updates (Alan Coopersmith).
				\end{itemize}
			\end{block}

			\begin{block}{X.Org developer guide}
				\begin{itemize}
					\item Written by Alan Coopersmith, Matt Dew and the X.Org team;
					\item Edited by Bart Massey;
					\item \url{http://www.x.org/wiki/guide/}.
				\end{itemize}
			\end{block}
		\end{frame}

	\section{Current situation}
		\subsection{Membership}
		\begin{frame}
			\begin{block}{Members}
				People actively engaged in activities related to X.Org, 
support the goal of the Foundation and who signed the membership agreement.
			\end{block}

			\begin{block}{Roles of the members}
				\begin{itemize}
					\item Elect the members of the board of
directors;
					\item Approve big changes proposed by the 
board (By-laws, corporate status, dissolution, unusual agreement, ...).
				\end{itemize}
			\end{block}

			\begin{block}{Current situation}
				\begin{itemize}
					\item 79 members!
					\item Join us:
\url{http://www.x.org/wiki/Membership}!
				\end{itemize}
			\end{block}
		\end{frame}

		\subsection{Members of the Board of Directors}
		\begin{frame}
			\begin{block}{Elected in 2012, for 2 years}
				\begin{itemize}
					\item Alex Deucher;
					\item Keith Packard;
					\item Matt Dew;
					\item Matthias Hopf.
				\end{itemize}
			\end{block}

			\begin{block}{Elected in 2013, for 2 years}
				\begin{itemize}
					\item Alan Coopersmith;
					\item Peter Hutterer (Secretary);
					\item Stuart Kreitman (Treasurer);
					\item Martin Peres.
				\end{itemize}
			\end{block}
		\end{frame}

		\begin{frame}
			\begin{block}{Re-election schedule, 4 slots (Martin)}
				\begin{itemize}
					\item Nomination period opens: 13/01/2014;
					\item Nomination period closes: 12/02/2014;
					\item Publication of candidates: 13/02/2014;
					\item Membership application deadline: 18/02/2014;
					\item Election period opens: 17/02/2014;
					\item Election period closes: 09/03/2014.
				\end{itemize}
			\end{block}

			\begin{block}{Be a candidate!}
				Apply as a candidate to increase your help to this great community!
			\end{block}

			\begin{block}{}
				\url{http://www.x.org/wiki/BoardOfDirectors/Elections/2014}
			\end{block}
		\end{frame}

		\subsection{501(c)(3)}
		\begin{frame}
			\begin{block}{Corporate Status}
				\begin{itemize}
					\item The Foundation has the 501(c)(3) status (not-for-profit);
					\item Allows for tax-deductible donations in the USA;
					\item Irrelevant for non-USA donators.
				\end{itemize}
			\end{block}

			\begin{block}{History of the Foundation's 501(c)(3) status}
				\begin{itemize}
					\item 2005: We deciced to apply for the 501(c)(3);
					\item 2012: The Foundation gained the status with the help
of the SFLC (Software Freedom Law Center);
					\item 2013: We lost the status because we didn't
fill a tax form (we had no income to declare and didn't get any info from SLFC);
					\item 2013: We got the status again after clearing up the issue.
				\end{itemize}
			\end{block}
		\end{frame}

		\begin{frame}
			\begin{block}{Complete summary of what happened}
				\url{http://who-t.blogspot.fr/2013/10/the-xorg-foundation-and-501c3-status.html}
			\end{block}
		\end{frame}

		\subsection{Bank}
		\begin{frame}
			\begin{block}{Current balance}
				At the current rate, we have 4 to 5 years left before requiring a fund raise.
			\end{block}

			\begin{block}{HSBC $\rightarrow$ Bank Of America (Stuart Kreitmann)}
				\begin{itemize}
					\item The Foundation has been an HSBC client;
					\item We've had a \$29/month ``analysis fee'' that wouldn't go away;
					\item Their web-portal wasn't particulary effective/helpful;
					\item HSBC changed rules and closed our checking account;
					\item We switched to Bank of America which for a better service.
				\end{itemize}
			\end{block}
		\end{frame}

	\section{On-going projects}
		\subsection{SPI merge}
		\begin{frame}
			\begin{block}{Merging with SPI}
				\begin{itemize}
					\item Keeping the 501(c)(3) status is a lot of work;
					\item It is work we aren't qualified to do (no accountants, not all of us are in the US);
					\item We don't get as much money as we used to, to justify paying for an accountant;
					\item There are umbrella corporations to help with that(SFLC, SPI);
					\item The board unanimously voted at XDC for SPI;
					\item We'll be contacting them and SLFC before requiring a members vote (Keith).
				\end{itemize}
			\end{block}
		\end{frame}

		\begin{frame}
			\begin{block}{Software in the Public Interest (SPI)}
				\begin{itemize}
					\item sets up donation infrastructures;
					\item manages our money and deals with the bank/state for us;
					\item keeps 5\% of all donations in return;
					\item handles ArchLinux, Debian, FreeDesktop, LibreOffice, etc...
				\end{itemize}
			\end{block}
		\end{frame}

		\subsection{Updating the ByLaws}
		\begin{frame}
			\begin{block}{Updating the By-laws}
				\begin{itemize}
					\item Merging with SPI requires changing the By-laws;
					\item They have been ported from OpenOffice to Latex/Git to help with
reviewing the changes and improve versionning (Martin);
					\item It is also the occasion to account for 
Mesa/DRI and Wayland in the By-laws and make them really a part of the scope of the Foundation.
				\end{itemize}
			\end{block}

			\begin{block}{By-laws repo}
				\url{http://anongit.freedesktop.org/git/xorg/foundation/bylaws.git}
			\end{block}
		\end{frame}

	\section{Conclusion}
		\begin{frame}
			\begin{block}{Conclusion}
				\begin{itemize}
					\item Become a member now, we'll need your opinion!
					\item Become a board candidate!
					\item Get involved and improve the communication!
					\item Keep on coding and improving the graphics stack ;)
				\end{itemize}
			\end{block}
		\end{frame}
% 		great. So, I'll be speaking to FOSDEM about the state of the
% foundation, what we are doing to it (SPI, change the scope in the bylaws) and
% asking people to become members, present themselves as candidate for the board
% and vote
% I'll start with basically what you said at XDC, whot
% I'll try to write the slides soon so as I can get them approved
% just to make sure we are all on the same page

\end{document}
