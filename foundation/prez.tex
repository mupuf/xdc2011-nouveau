\documentclass{beamer}
\usepackage[utf8]{inputenc}
\usepackage[english]{babel}
\usetheme{Antibes}

% À chaque nouvelle section, on dit où on est
\AtBeginSection[ ]
{
 \begin{frame}<beamer>
   \frametitle{Summary}
   \tableofcontents[currentsection]
  \end{frame}
} 

\title{State of the X.Org Foundation}
\subtitle{FOSDEM 2014}
\author{Martin~Peres}
\institute{\includegraphics[height=1.0cm]{figures/logo.png}}

\begin{document}
	\begin{frame}
		\titlepage
	\end{frame}

	\logo{\includegraphics[height=1.0cm]{figures/logo_small_200px.png}}

	\section{Introduction}
		\subsection{Role of the Foundation}
		\begin{frame}
			\begin{block}{}
				The X.Org Foundation is a company chartered to
develop and execute effective strategies that provide worldwide stewardship and
encouragement of the X Window System and related projects (Mesa, DRI, Wayland,
etc.).
			\end{block}

			\begin{block}{}
				\url{http://www.x.org/wiki/XorgFoundation/}
			\end{block}

			\begin{center}
				\includegraphics[width=.15\linewidth]{figures/logo_small_200px.png}
				\hspace{.05\linewidth}
				\includegraphics[width=.15\linewidth]{figures/xcb_logo_small.png}
				\hspace{.05\linewidth}
				\includegraphics[width=.15\linewidth]{figures/tux_small.jpg}
				\hspace{.05\linewidth}
				\includegraphics[width=.15\linewidth]{figures/glxgears-trans.png}
				\hspace{.05\linewidth}
				\includegraphics[width=.15\linewidth]{figures/wayland_logo_small.png}
			\end{center}

		\end{frame}

		\subsection{Actions of the Foundation}
		\begin{frame}
			\begin{block}{X.Org Developer's Conference (XDC/XDS)}
				\begin{itemize}
					\item Organized once a year, around
September/October;
					\item Alternates between the US and
Europe (in Europe this year!);
					\item Lasts 3 days;
				\end{itemize}
			\end{block}

			\begin{block}{Travel sponsorship}
				The board of directors can cover the travel and
accomodation expenses to the developers who couldn't attend an X.org-related
conference (X.org or FOSDEM?) otherwise.
			\end{block}
		\end{frame}

		\begin{frame}
			\begin{block}{Google Summer of Code (GSoC)}
				\begin{itemize}
					\item The X.org Foundation is an
approved org. of the GSoC;
					\item This allows students to participate to
X.org-related projects over the summer and get paid to flip bits, not burgers.
				\end{itemize}
			\end{block}

			\begin{block}{GSoC projects of 2013}
				\begin{itemize}
					\item David Hermann (Dave Airlie):
Render Nodes and security fixes
					\item Samuel Pitoiset (Martin Peres):
Reverse engineering performance counters on NVIDIA cards and exposing Fermi MP
counters in Mesa;
					\item Dylan Noblesmith (Ian Romanick):
Implementing GL\_EXT\_direct\_state\_access.
				\end{itemize}
			\end{block}
		\end{frame}

		\begin{frame}
			\begin{block}{Endless Vacation of Code (EVoC)}
				\begin{itemize}
					\item EVoC is a GSoC-like project, funded by the X.org Foundation;
					\item It allows students to participate to
X.org-related projects during their vacation, at anytime of the year.
				\end{itemize}
			\end{block}

			\begin{block}{EVoC projects}
				\begin{itemize}
					\item 2013: No projects;
					\item 2012:
					\begin{itemize}
						\item Francisco Jerrez (None): Gallium/Nouveau - OpenCL support
						\item Supreet Pal Singh (Martin Peres): Nouveau - Software scripting Engine for Fermi Architecture based GPUs;
						\item Blaž Tomažič (?): OpenCL Testing Framework for Piglit. Project Overview 
						\item And more...
					\end{itemize}
				\end{itemize}
			\end{block}
		\end{frame}

		\begin{frame}
			\begin{block}{Communication: Google+ and Twitter}
				\begin{itemize}
					\item Google+/Youtube: Store and/or link
to talk videos, slides, blog articles or G+ posts related to projects under our umbrella (Martin Peres);
					\item Twitter: Mostly security issues, random updates (Alan Coopersmith).
				\end{itemize}
			\end{block}

			\begin{block}{X.org developer guide}
				\begin{itemize}
					\item Written by Alan Coopersmith, Matt Dew and the X.org team;
					\item Edited by Bart Massey;
					\item \url{http://www.x.org/wiki/guide/}.
				\end{itemize}
			\end{block}
		\end{frame}

	\section{Current situation}
		\subsection{Membership}
		\begin{frame}
			\begin{block}{Members}
				People actively engaged in activities related to X.org, 
support the goal of the Foundation and who signed the membership agreement.
			\end{block}

			\begin{block}{Roles of the members}
				\begin{itemize}
					\item Elect the members of the board of
directors;
					\item Approve big changes proposed by the 
board (By-laws, corporate status, dissolution, unusual agreement, ...).
				\end{itemize}
			\end{block}

			\begin{block}{Current situation}
				\begin{itemize}
					\item 79 members!
					\item Join us:
\url{http://www.x.org/wiki/Membership}!
				\end{itemize}
			\end{block}
		\end{frame}

		\subsection{Members of the board of direction}
		\begin{frame}
			\begin{block}{Elected in 2012, for 2 years}
				\begin{itemize}
					\item Alan Coopersmith;
					\item Alex Deucher;
					\item Matt Dew;
					\item Matthias Hopf.
				\end{itemize}
			\end{block}

			\begin{block}{Elected in 2013, for 2 years}
				\begin{itemize}
					\item Peter Hutterer (Secretary);
					\item Stuart Kreitmann (Treasurer);
					\item Keith Packard;
					\item Martin Peres.
				\end{itemize}
			\end{block}
		\end{frame}

		\begin{frame}
			\begin{block}{Re-election schedule, 4 slots}
				\begin{itemize}
					\item Nomination period opens: 13/01/2014;
					\item Nomination period closes: 12/02/2014;
					\item Publication of candidates: 13/02/2014;
					\item Membership application deadline: 18/02/2014;
					\item Election period opens: 17/02/2014;
					\item Election period closes: 09/03/2014.
				\end{itemize}
			\end{block}

			\begin{block}{}
				\url{http://www.x.org/wiki/BoardOfDirectors/Elections/2014}
			\end{block}
		\end{frame}

		\subsection{501(c)(3)}

		\subsection{Banks}

		% problems with transfering funds (~100€ per person got lost):
% HSBC -> Bank Of America
		%  $29/month and up (depending on activity) "analysis fee"
% 		HSBC closed the checking account unilateraly

	\section{Current projects}
		\subsection{SPI merge}
		\subsection{Updating the ByLaws}


% 		great. So, I'll be speaking to FOSDEM about the state of the
% foundation, what we are doing to it (SPI, change the scope in the bylaws) and
% asking people to become members, present themselves as candidate for the board
% and vote
% I'll start with basically what you said at XDC, whot
% I'll try to write the slides soon so as I can get them approved
% just to make sure we are all on the same page

\end{document}
